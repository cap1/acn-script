\section{Lecture 3 - Online Social Networks} % (fold)
\label{sec:lecture_3_online_social_networks}

\subsection{Introduction} % (fold)
\label{sub:introduction}
Online Social Networks (OSN) are large online communities like Facebook and Twitter.
These are the basic terms in OSN:\\
\begin{description}
	\item[Social Network] \hfill \\
	A network of a set of individuals,
	connecting with each other on the basis of social relationships.
	\item[Entity] The basic unit of the network
	\item[Link] Interconnection between entities
	\item[Behavior \& Dynamics] \hfill \\
	The implicit consequences of the actions done by the entities.
	There is no isolation and small actions can lead to large effects.
\end{description}

Online Social Networks share some features with other networks.
At first glance they seem random,
bit they display signatures of order and \emph{self-organization}.
The characteristics are:\\
\begin{description}
	\item[Virtual] does not exist physically
	\item[Complex] consists of large number of nodes
	\item[Grouping] forms communities due to different similarities or interest
	\item[Dynamic] structure evolves in time
\end{description}
% subsection introduction (end)

\paragraph{The analysis of OSN} % (fold)
\label{par:the_analysis_of_osn}
can be done using different sets of methods.
Studying a network empirical, one can find the organization principles of the network.
With mathematical models on leverages probabilistic and graph theory
and there are also dedicated algorithms that help understanding  and analyzing networks.

The general targets one tries to achieve are:\\
\begin{itemize}
	\item patterns and statistical properties of network data
	\item desgin principles and models
	\item understand why OSNs are organized the way they are (prediction)
\end{itemize}
% paragraph the_analysis_of_osn (end)

\subsection{Modeling Social Networks} % (fold)
\label{sub:modeling_social_networks}
Networks can be represented as a mathematical graph.
A \emph{Node} is then an object on the network,
that can be connected via an \emph{Edge} to other nodes.

Depending on the graph the edges can be directed (Following on twitter)
or indirected (Friendship on Facebook).
A \emph{Path} then describess a sequence of interconnected nodes
and a \emph{Circle} is a Path where the first and the last nodes are the same.
The \emph{Connectivity} of a graph defines how well the nodes are interconnected.
A connected graph contains only pairs of nodes,
who have a path in between them.
If the graph is not connected,
it can be broken into several connected subgraphs.
A \emph{Connected Component} is a subset of nodes,
where every node in the subset has a path to every other node
and the subset ist not part of a larger subset with the same property.
% subsection modeling_social_networks (end)
% subsection subsection_name (end)

% section lecture_3_online_social_networks (end)
