\section{Lecture 3 - Online Social Networks} % (fold)
\label{sec:lecture_3_online_social_networks}

Intro\\
Online Social Networks (OSN) are large online communities like Facebook and Twitter.
These are the baisc terms in OSN:\\
\begin{description}
	\item[Social Network] \hfill \\
	A network of a set of individuals,
	connecting with each other on the basis of social relationships.
	\item[Entity] The basic unit of the network
	\item[Link] Interconnection between entities
	\item[Behavior \& Dynamics] \hfill \\
	The implicit consequences of the actions done by the entities.
	There is no isolation and small actions can lead to large effects.
\end{description}

Online Social Networks share some features with other networks.
At first glance they seem random,
bit they display signatures of order and \emph{self-organization}.
The characteristics are:\\
\begin{description}
	\item[Virtual] does not exist physically
	\item[Complex] consists of large number of nodes
	\item[Grouping] forms communities due to different similarites or interest
	\item[Dynamic] structure evolves in time
\end{description}

% section lecture_3_online_social_networks (end)
