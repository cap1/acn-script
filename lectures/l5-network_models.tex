\section{Lecture 5 - Network Models} % (fold)
\label{sec:lecture_5_network_models}
As social networks play an important role on its inhabitants.
They influence the flow of information,
the spread of disease,
the trade of goods and services as well as which languages we speak and what we vote.\\
With network analysis technologies we can try to understand,
how their structure influences us
and which structures are likely to emerge in the society.

\paragraph{Marriage Network in Florentine} % (fold)
\label{par:marriage_network_in_florentine}
	An example can be the network of marriages in Florentine in the fifteenth century.
	Through network analysis one can find hints for the shift of power from the 
	Strozzi family to the Medici family.
	The Degree Centrality (\ref{ssub:degree_centrality}) for the Strozzi family in the network
	was only 4/15, but for the Medicis it was 6/15.
	The Betweenness Centrality of the Strozzis was $0.103$,
	which is very low compared to the $05.22$ of the Medicis,
	as more than half of the shortest paths between influential families went through the Medicis.
	This not only illustrates the importance of the families,
	but shows that knowledge about network structure gives insights into political and economic structures.
% paragraph marriage_network_in_florentine (end)

\subsection[Random Network]{Random Network - Erd\"os-Renyi\\} % (fold)
\label{ssub:erdoes-renyi}
These networks are generated in a random process,
whom origins with fixed set of nodes $N = {1,2,..,n}$.
Then links between the nodes are added with a given probability $p (0 < p < 1)$.\\
For a given network with $m$ links,
the probability that it is formed is:

\begin{equation}
\label{eq:erdoesrenyiprob}
	p^m (1 - p)^{\frac{n(n-1)}{2}-m}
\end{equation}
The Degree Distribution (\ref{ssub:degree_distribution})
that any given node $i$ has exactly $d$ links is then

\begin{equation}
\label{eq:erdoesrenyidegreedist}
	Pr(d) = \binom{n-1}{d} p^d(1-p)^{n-1-d}
\end{equation}
For large $n$ and small $p$,
this binomial expression is approximated by a Poisson distribution

\begin{equation}
\label{eq:erdoesrenyipoisson}
	Pr(d) = \frac{e^{-(n-1)p}((n-1)p)^d}{d!}
\end{equation}
Such networks are called \emph{Poisson Random Networks}.
The connectivity of the graph can be calculated easily with Equation \ref{eq:erdoesrenyipoisson}.

\begin{equation}
	Pr(0) = e^{-(n-1)p}
\end{equation}
One can further solve this and when:

\begin{equation}
	(n-1)p= log (n)
\end{equation}
is larger than $p$ the graph will break in several components.
If $p > (n-1)p= log (n)$ the graph is connected with a high probability.
% subsubsection erdoesrenyi (end)

\subsection[Small World Network]{Small World Network - Watts-Strogatz} % (fold)
\label{sub:watts_strogatz}
This Model tries to generate a network with a low diameter (\ref{par:diameter})
and a high clustering coefficient (\ref{ssub:clustering_coefficient}).
The generation starts with a regular lattice
and continuously rewires the network with a given probability $p | 0<p<1$.
With $p = 0$, a regular network will be generated
and with $p = 1$ a random one will emerge.
The randomization takes place by lowering the average path length (\ref{par:average_path_length}).

Networks created with \emph{Small World Model} have a high clustering coefficient
and a low diameter.
% subsection watts_strogatz (end)

\subsection{Scale-free Network} % (fold)
\label{sub:scale_free_network}
Scale-free Networks are networks whose degree distribution follows a Power law.
Several models are proposed to generate these networks randomly.
The simple \emph{Rich get richer model},
which will be explained here,
the preferential attachment by Barabasi 
and the Fitness Model by Biancomi and by Kong.

\paragraph{The Rich get richer model} % (fold)
\label{par:the_rich_get_richer_model}
follows the Power law.
In the process of creation new nodes are added to the network in each step.
With a given probability this node.
% paragraph the_rich_get_richer_model (end)
% subsection scale_free_network (end)


% section lecture_5_network_models (end)