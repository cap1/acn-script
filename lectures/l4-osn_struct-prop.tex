\section{Lecture 4 - Social Networks: Structure \& Properties} % (fold)
\label{sec:social_networks_structure_&_properties}

Networks can characterized in different ways.
Distinctive features are:\\
\begin{description}
	\item[Degree Distribution] How many Neighbors has a node?
	\item[Average Path Length] Whats the mean number of hops between nodes?
	\item[Network Diameter] How far apart are the nodes?
	\item[Distance - Shortest Path] How big is the network?
	\item[Clustering Coefficient] How close is a set of nodes connected with each other?
	\item[Community] Are subsets more closely connected?
\end{description}

\subsection{Network Properties} % (fold)
\label{sub:network_properties}
	The key Network properties are described in this Section,
	Degree Distribution, Path Length and Clustering Coefficient.

	\subsubsection{Degree Distribution} % (fold)
	\label{ssub:degree_distribution}
	The degree distribution $P(k)$ describes the probability of randomly given node,
	having the degree $k$.
	It can be calculated as described in Equation \ref{eq:degree_distribution}.

	\begin{equation}
	\label{eq:degree_distribution}
		P(k) = N_k / N
	\end{equation}

	Where $N_k$ is the number of nodes.
	% subsubsection degree_distribution (end)

	\subsubsection{Path Length} % (fold)
	\label{ssub:path_length}
		The Path Length subsumes several metrics for the network size.
		\paragraph{Distance} % (fold)
		\label{par:distance}
		describes the number of edges across the shortest path on the network.
		This also means that if two nods are not connected,
		their distance is infinite.
		% paragraph distance (end)

		\paragraph{Diameter} % (fold)
		\label{par:diameter}
		is the maximum distance in terms of edges between any nodes of the network.
		It is the opposite metric to the distance.
		% paragraph diameter (end)

		\paragraph{Average Path Length} % (fold)
		\label{par:average_path_length}
		is the combined measurement of the distance and the diameter,
		taking the average distance between all the nodes in the network.
		% paragraph average_path_length (end)
	% subsubsection path_length (end)

	\subsubsection{Clustering Coefficient} % (fold)
	\label{ssub:clustering_coefficient}
		This metric evaluated how the neighbors of a node are interconnected.
		For a given node $i$ with the degree $k$,
		assume the number of neighbors of $i$ is $e$,
		then the clustering coefficient of $i$ is described in equation \ref{eq:clustercoefficient}.
		\begin{equation}
		\label{eq:clustercoefficient}
		C_i - \frac{e}{k(k - 1)/2}
		\end{equation}

		The average Clustering Coefficient for network is then:
		\begin{equation}
		\label{eq:averagecc}
		C = \frac{1}{N} \sideset{}{_i^N}\sum C_i
		\end{equation}
	% subsubsection clustering_coefficient (end)

	\subsubsection{Degree Centrality} % (fold)
	\label{ssub:degree_centrality}
		This measurement determines how connected is a certain node inside the network.
		\begin{equation}
		\label{eq:degreecentrality}
			d_i(g) / (n-g)
		\end{equation}
	% subsubsection degree_centrality (end)

	\subsubsection{Closeness Centrality} % (fold)
	\label{ssub:closeness_centrality}
		Given the Closeness Centrality one can determine,
		how easily a node can reach other nodes in the network.
		\begin{equation}
		\label{eq:closenesscentrality}
			(n-1) / \sum_{j\neq i} l(i,j)
		\end{equation}
		Where $l(i,j)$ is the length of the shortest path between the nodes i and j.
	% subsubsection closeness_centrality (end)

	\subsubsection{Betweenness Centrality} % (fold)
	\label{ssub:betweenness_centrality}
		The Betweenness Centrality of a node gives a measurement how important a node is,
		in terms of connecting other nodes of the network.
		\begin{equation}
		\label{eq:betweennesscentrality}
			Ce_i^B (g) = \sum_{k\neq j:i\notin\{k,j\}} \frac{P_i(kj)/P(kj)}{(n-1)(n-2)/2}
		\end{equation}	
	% subsubsection betweenness_centrality (end)
% subsection network_properties (end)

\subsection{Metrics of typical networks} % (fold)
\label{sub:metrics_of_typical_networks}

	\subsubsection{Complete Graph} % (fold)
	\label{ssub:complete_graph}
	All nodes are interconnected.
		\begin{description}
			\item[Degree Distribution] $P(k) = N-1$
			\item[Path Length] Diameter = 1 --- Average Path Length = 1
			\item[Clustering Coefficient] C = 1 --- Average Clustering Coefficient = 1
		\end{description}
	% subsubsection complete_graph (end)

	\subsubsection*{Regular Latice} % (fold)
	\label{ssub:regular_latice}
	A Ring connecting only neighbors.
	Each node is also connected to its successors and its predecessors neighbor.
	\begin{description}
			\item[Degree Distribution] $P(k) = N-1$
			\item[Path Length] Diameter = 1 --- Average Path Length = 1
			\item[Clustering Coefficient] C = 1 --- Average Clustering Coefficient = 1
		\end{description}
	% subsubsection regular_latice (end)

	\subsubsection{Random Graph} % (fold)
	\label{ssub:random_graph}
	This depends on how the random graph is generated.
	Here we assume a binomial distribution.
		\begin{description}
			\item[Degree Distribution] $P(k) = \binom{n-1}{k}p^k(1-p)^{n-1-k}$
			\item[Average Path Length] $O(\log n)$
			\item[Clustering Coefficient] $C = p = \bar{k}/n$
		\end{description}
	% subsubsection random_graph (end)
% subsection metrics_of_typical_networks (end)

\subsection{Power Law} % (fold)
\label{sub:power_law}
When observing the popularity of nodes in a network,
one can see that there is an imbalance.
Some nodes have far higher degrees and receive more connections than others.
A typical example is that 20\% of the websites,
receive 80\% of the traffic.
This effect can be observed in many other networks like twitter,
where celebrities have a huge amount of followers,
or the imbalance of money distribution,
where only a few very rich people own a large amount.

The degree distribution is not a Gaussian one,
but a Power Law distribution.
% subsection power_law (end)

% section social_networks_structure_&_properties (end)