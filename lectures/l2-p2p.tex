\section{Lecture 2 - Introduction to P2P-Networks} % (fold)
\label{sec:lecture_2_introduction_to_p2p_networks}
Peer-to-Peer (P2P) Networks lack a central server so that all entities are equal in the network.

\subsection{Features} % (fold)
\label{sub:features}
\begin{description}
	\item[Decentralized] \hfill \\
		No central server or component, all Peers are equal
	\item[Self-organized] \hfill \\
		Dynamic behavior of nodes, free to join, part, quit
	\item[Overlay Network] \hfill \\
		Leverages the Internet and uses a flat architecture
	\item[Large Scale resource] \hfill \\
		Heterogeneous network with large possible number of nodes
	\item[Collaboration] \hfill \\
		Based on voluntary participation with global reach
	\item[Flexible] \hfill \\
		Resilient to attacks and anonymous
\end{description}
% subsection features (end)

\subsection{Problems with the Client/Server Model} % (fold)
\label{sub:problems_with_the_client_server_model}
The amount of Data the normal users transmit has been growing massively over last years.
Due to the increase in bandwidth,
users can also transmit these amounts of data.

\paragraph{Client/Server} % (fold)
\label{par:client_server}
is the classical model for the Internet.
A single server responds to the request of all users.
This server is also the problem of the model.
It may be easier to manage the single server,
but it is also the bottleneck and a single-point of failure.
If this server breaks down,
the service will not be available
and a single server just won't scale.
% paragraph client_server (end)

\paragraph{Replication} % (fold)
\label{par:replication}
solves some problems with the client/server model.
The server is mirrored so that it is not a single-point of failure anymore.
Although replication will still not scale well
and one can run into trouble due to high-workload on a single replica.
This single replica than has the same problems as a single server
as the workload is not distributed among the others.
Also maintenance might be a problem.
% paragraph replication (end)

\paragraph{Proxy \& CDN} % (fold)
\label{par:proxy_&_cdn}
can circumvent the problems with balancing the workload,
but still do not scale well in terms of bandwidth and CPU power.
And as they increase the complexity,
they maintenance gets more difficult.
% paragraph proxy_&_cdn (end)
% subsection problems_with_the_client_server_model (end)

\subsection{Peer 2 Peer} % (fold)
\label{sub:peer_2_peer}
Peer-to-Peer has some advantages over the classical client/server-model.
These systems work even better if more users are connected.
They are designed to be easy to use and to deploy,
to be fault tolerant.
Users can connect and leave dynamically and the quality of service is not altered.

\paragraph{Current State}
of P2P networks is that their overall share of network bandwidth used is declining.
This is due to the fact that the overall amount of bandwidth used is increasing,
while P2P traffic is not increasing.

Topics in research are the structure of P2P Networks,
regarding structured and unstructured P2P.
The security and the privacy of the users are also a research topic,
as well as the legal issues associated with P2P
Locating the the desired data in the P2P networks has always been an interesting thing,
as different types of networks solve this in different ways.

\paragraph{Structured \& Unstructured} % (fold)
\label{par:structured_unstructured}
are the main groups to sort P2P networks.
Unstructured networks are highly flexible,
dynamic and easy to maintain.
They base on the loosely connected nodes having no central point.
Due to this, it is hard to look information up.

Structured P2P networks have a fixed structure,
which enables an easy look up to find information.
% paragraph structured_unstructured (end)

\subsubsection{BitTorrent} % (fold)
\label{ssub:bittorrent}
BitTorrent is an unstructured P2P Network, that is very popular.
It accounted for 37\% of the Internet traffic in the German Internet in 2009.
The basic concepts of BitTorrent are \emph{chunking} and \emph{swarming}.
Where the files are split into smaller pieces or chunks,
so they can be downloaded in parallel and the order of download becomes irrelevant.
The chunks are then transfered between the nodes in a swarm,
who exchange the same content.

The basic components in a BitTorrent network are:
\begin{description}
	\item[Tracker] special server, tracking active peers and mapping file names to them
	\item[Peer] downloading as a leecher or sharing complete copy as seeder
	\item[.torrent] file containing the meta data of the data, ie. trackers IP and hashes of chunks
\end{description}

BitTorrent enforces a Tit-for-Tat policy,
so that users are encouraged to share bandwidth and data.
Also peers serve peers who that serve them,
so that free-riding is discouraged.
The chunk selection is done in a rares-first policy.
This maximizes the availability of each chunk.

BitTorrent is an efficient P2P protocol,
as it is fast, resilient, maximizes the availability of the data
and discourages free-riding.
Problems are that the tracker could be a single point of failure
and the lack of a search feature.
% subsubsection bittorrent (end)
% subsection peer_2_peer (end)

\subsection{Structured Peer 2 Peer Networks} % (fold)
\label{sub:structured_peer_2_peer_networks}
The common problem in P2P networks is how to implement the search for data in the network.
The requirements for the lookup of data in the network are as follows:
\begin{description}
	\item[Scalable] operating with millions of nodes
	\item[Self-organized] no central or external control
	\item[Load-distribution] all participants shall contributed
	\item[Fault-tolerant] robust against leaves and fails of nodes
	\item[Robustness] resilient against malicious nodes 
\end{description}

\subsubsection{Distributed Hash Tables} % (fold)
\label{ssub:distributed_hash_tables}
The basic idea of Distributed Hash Tables (DHT) is to compute the hash value of the data
and the hash value of the IP addresses.
The content, or a content locator,
is then stored at the machine with the closest hash value.

The method has been introduced in 2001 in four Papers referring to the networks
\emph{CAN},  \emph{Chord}, \emph{Pastry} and \emph{Tapestry}.
\emph{BitTorrent}\ref{ssub:bittorrent} also makes use of Kadamelia DHT.

	\paragraph{Hash Functions} % (fold)
	\label{par:hash_functions}
	map an arbitrary input sequence to an output with a fixed length.
	A slight change in the input usually leads to a huge change in the output
	as the functions span the whole 2\textsuperscript{k} space of the hashlength \texttt{k}. 

	Typical has functions are \texttt{SHA-1} or \texttt{MD5},
	they usually differ in certain aspects like possibility of collisions and the hash value length.
	% paragraph hash_functions (end)

	\paragraph{Chord} % (fold)
	\label{par:chord}
	can serve as an example for the implementation of DHT.
	Given the identifier space is 2\textsuperscript{4} (0 .. 15).
	The nodes also have IDs in this space,
	which is not filled up.
	So the nodes just cover a subset of the identifiers available (2,5,6,11,14).
	The have a pointer to their successor,
	the node with a larger or equal ID in the clockwise direction.
	Each key, value pair of data is assigned the identifier of Hash(key).
	The item is then stored at its Successor(Hash(key)).

	The basic lookup can now be achieved in \textit{O(n)},
	as in the worst case one has to traverse one time over whole ring.
	This can be optimized to \textit{O(log n)} by adding a finger table to each node.
	This finger table contains fingers for $$succ(n+2^{i-1}$$ as shortcuts through the ring of nodes
	and thereby reduces the time to route to entries to \textit{$$log_2$$(n)}.

	Although each node just has a limited view of the network,
	the concept of numeric closeness solves the problems with routing and lookup.
	
	The lookup can be done iteratively,
	so that the requesting client keeps asking the next node where to continue looking.
	It can also be implemented recursively,
	so that the node querried first,
	will keep asking its successor and finally answer the request.
	Recursuve lookup allows each node to optimize its request
	and allows fast and efficient processing of the request.
	Iterative lookup allows the client to keep control
	and it can detect failures and localize them.
	% paragraph chord (end)

% subsubsection distributed_hash_tables (end)
% subsection structured_peer_2_peer_networks (end)

% section lecture_2_introduction_to_p2p_networks (end)
