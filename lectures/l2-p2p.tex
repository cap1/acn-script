\section{Lecture 2 - Introduction to P2P-Networks} % (fold)
\label{sec:lecture_2_introduction_to_p2p_networks}
Peer-to-Peer (P2P) Networks lack a central server so that all entities are equal in the network.

\subsection{Features} % (fold)
\label{sub:features}
\begin{description}
	\item[Decentralized] \hfill \\
		No central server or component, all Peers are equal
	\item[Self-organized] \hfill \\
		Dynamic behavior of nodes, free to join, part, quit
	\item[Overlay Network] \hfill \\
		Leverages the Internet and uses a flat architecture
	\item[Large Scale resource] \hfill \\
		Heterogeneous network with large possible number of nodes
	\item[Collaboration] \hfill \\
		Based on voluntary participation with global reach
	\item[Flexible] \hfill \\
		Resilient to attacks and anonymous
\end{description}
% subsection features (end)

\subsection{Problems with the Client/Server Model} % (fold)
\label{sub:problems_with_the_client_server_model}
The amount of Data the normal users transmit has been growing massively over last years.
Due to the increase in bandwidth,
users can also transmit these amounts of data.

\paragraph{Client/Server} % (fold)
\label{par:client_server}
is the classical model for the Internet.
A single server responds to the request of all users.
This server is also the problem of the model.
It may be easier to manage the single server,
but it is also the bottleneck and a single-point of failure.
If this server breaks down,
the service will not be available
and a single server just won't scale.
% paragraph client_server (end)

\paragraph{Replication} % (fold)
\label{par:replication}
solves some problems with the client/server model.
The server is mirrored so that it is not a single-point of failure anymore.
Although replication will still not scale well
and one can run into trouble due to high-workload on a single replica.
This single replica than has the same problems as a single server
as the workload is not distributed among the others.
Also maintenance might be a problem.
% paragraph replication (end)

\paragraph{Proxy \& CDN} % (fold)
\label{par:proxy_&_cdn}
can circumvent the problems with balacing the workload,
but still do not scale well in terms of bandwidth and CPU power.
And as they increase the complexity,
they maintenance gets more difficutlt.
% paragraph proxy_&_cdn (end)
% subsection problems_with_the_client_server_model (end)

\subsection{Peer 2 Peer} % (fold)
\label{sub:peer_2_peer}
Peer-to-Peer has some advantages over the classical client/server-model.
These systems work even better if more users are connected.
They are designed to be easy to use and to deploy,
to be fault tolerant.
Users can connect and leave dynamically and the quality of service is not altered.

\paragraph{BitTorrent} % (fold)
\label{par:bittorrent}
The basic ideas in P2P-Networks are \emph{chunking} and \emph{swarming}.
To enhance the download speed,
the files are split in smaller chunks.
These can be downloaded in parallel
and in any given order.

Clients join a crowd of peers for the same content.
They request chunks from their neighbors
and download content in parallel.

A central node is used to publish the content
and find peers to get content.
This \emph{Tracker}, is a webserver that tracks the active peers
and maps them.
The \emph{Peers} can be distinguished in Seeders and Leechers,
where the later do not have a complete copy of the file.
In the \emph{.torrent} file the file to be shared among the peers is described with its metadata.
It contains:\\
\begin{itemize}
	\item Number of chunks
	\item Trackers IP
	\item File Metadata (Checksums)
\end{itemize}


% paragraph bitto (end)

% paragraph bittorrent (end)
% subsection peer_2_peer (end)

% section lecture_2_introduction_to_p2p_networks (end)