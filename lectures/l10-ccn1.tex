\section{Lecture 10 \& 11 - Content Centric Networks} % (fold)
\label{sec:lecture_10_content_centric_networks}
A Content Centric Network (CCN) is a so called Next Generation Network.
It is a concept that focuses on retrieving the data that is requested by the users
instead of letting the user reference a certain physical location where the data is stored.\\
The Current Network relies on a host based architecture that was created in the 1970s.

\subsection{Creating Content Centric Networks} % (fold)
\label{sub:creating_content_centric_networks}
The basic idea is to replace the IP Protocol,
with a \emph{Content Name}.
It consists of several parts that allow addressing content
and looks like an URL.
It starts with a globally routable name
who is followed by user, appsupplied or organizational name.
The last part can determine a certain version or segment of the data.

The data is send in \emph{Data Packets},
who are send in a query-response fashion.
The data packets are cached at each level of the network.
This caching allows improvements in speed of serving the data,
as the next time it can be served directly from the cache.
In the worst case its is like the current service,
as the data is request directly form its source.

\paragraph{Interests} % (fold)
\label{par:interests}
are the data packets that are send out to request data.
They contain the Content Name as the identifier,
a selector that determines order preference, scope or a certain filter
and finally a nonce that will allow cryptographically secure answers.
% paragraph interests (end()
\paragraph{Data} % (fold)
\label{par:data}
are the responses to interests and contain the data.
Additionally they contain the Content Name,
an optional description,
a signature from a digest algorithm,
additional signed information from the publisher
and the data itself.
% paragraph data (end)

% subsection creating_content_centric_networks (end)
% section lecture_10_content_centric_networks_1 (end)