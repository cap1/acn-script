\section{Lecture 9 - Scaling of OSN} % (fold)
\label{sec:lecture_9_scaling_of_osn}

\subsection{Introduction to OSN Scaling}
As Online Social Networks are extremely popular,
their demand for a scalable infrastructure to support their growth is natural.
Cloud COmputing can provide such scaling,
but it leads to some problems that need to be dealt with.

Conventional scaling approaches are horizontal or vertical.
Where \emph{vertical} stands for the the upgrading the existing hardware.
This can be expensive and be technically infeasible.
\emph{Horizontal} scaling means deploying a larger number of servers
and distributing the workload amongst them.
Although this is only suitable for stateless front-end servers,
on back-end storage servers the data must be partitioned into disjoint components.

Bot approaches are not applicable to OSNs,
as the large amount of data rules out vertical scaling 
and the large amount of interconnect makes horizontal scaling infeasible.
This refers to the fact that in an OSN,
most operations are between users and their neighbors.
Their data is usually placed on multiple servers
and the mu 'multi-get' operations generate a huge amount of traffic
and unpredictable response times.

\subsection{Social Paritioning and Replication} % (fold)
\label{sub:social_paritioning_and_replication}
A novel solution for these problems could be \emph{SPAR} (Social Partitioning And Replication).
Its approach is to have a one-hop replication
so that all neighbors of users are replicated on the same server.
This leads to a 'Social Locality'.\\
The requirements for SPAR are:
	\begin{itemize}
		\item Maintain local semantics
		\item Balance loads
		\item Be resilient to machine failures
		\item Be amenable to online operations
		\item Be stable
		\item Minimize the replication overhead
	\end{itemize}
% subsection social_paritioning_and_replication (end)

\subsection{OSN in the Cloud} % (fold)
\label{sub:osn_in_the_cloud}
OSNs require features that can be delivered very well with cloud computing.
The networks need to be available in a reliable fashion
on a global scale.
They also require high amounts of flexibility in the amount of workload they can do.

Cloud computing can provide the availability, flexibility and scalability required by OSNs.
Due to the 'pay-as-you-go' plans the cost do not have to paid in front.
The cloud allows to serve customers,
but keep the cost at a low level.
Although the Quality of Service (QoS) must be maintained.
% subsection osn_in_the_cloud (end)
% section lecture_9_scaling_of_osn (end)