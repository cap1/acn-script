\section{Lecture 9 - Scaling of OSN} % (fold)
\label{sec:lecture_9_scaling_of_osn}

\subsection{Introduction to OSN Scaling}
As Online Social Networks are extremly popular,
their demand for a scalable ifrastructure to support their grwoth is natural.
Cloud COmputing can provide such scaling,
but it leads to some problems that need to be deatl with.

Conventional scaling approaches are horizontal or vertical.
Where \emph{vertical} stands for the the upgrading the existing hardware.
This can be expensive and be techincally infeasible.
\emph{Horizontal} scaling means deploying a larger number of servers
and distributing the workload amongst them.
Although this is only suitable for stateless frontend servers,
on backend storage servers the data must be partitioned into disjoint components.

Bot approaches are not applicable to OSNs,
as the large amaout of data rules out vertical sacling 
and the large amount of interconnect makes horizontal scaling infeasible.
This refers to the fact that in an OSN,
ost operations are between users and their neighbors.
Their data is usually placed on multiple servers
and the mu 'multi-get' operations genrate a huge amount of traffic
and unpredictable responsetimes.
% section lecture_9_scaling_of_osn (end)
