\section{Lecture 6 - Information Cascade \& Influence Maximization} % (fold)
\label{sec:lecture_6_information_cascade_&_influence_maximization}

This section focuses on the influence of information on the behavior
and the decisions of people.

\subsection{Information Cascades} % (fold)
\label{sub:information_cascades}
Information cascades may occur when people make decisions sequentially,
like when later people watch the actions of earlier people
and then inferring something about what the earlier people know.
A cascade then develops when the people abandon their own information
in favor of inferences based on other peoples actions.
The later individuals imitate the behavior of other,
from inferences of limited information.

Milgram performed a second famous experiment,
where a group of people where standing at a lively street corner
and started watching into the sky.
Depending on the amount of people starring at the sky,
more people joined the group.
If 15 people were looking upwards,
45\% of the passersby would start looking up too.

\paragraph{Herding} % (fold)
\label{par:herding}
is the information cascade in the previous example.
It requires a decision to be made,
that can be taken out sequentially.
The decision is guided by some private information,
which cannot be observed by others.
They can only see the actions resulting from this private information.
% paragraph herding (end)

\paragraph{Effects} % (fold)
\label{par:effects}
Information Cascades and Rich-get-richer-models explain some parts of social behavior.
This can be observed in Rumors,
adaption of new technologies in a peer group,
fashion and even in stock markets.
% paragraph effects (end)
% subsection information_cascades (end)

\subsection{Social Influence Maximization} % (fold)
\label{sub:social_influence_maximization}
Networks consist in this context of nodes (people)
and edges (social interactions, ties).

The behavior of people is influenced by the behavior of other people.
Understanding this influence is interesting for several aspects
like rumours, collective dynamics and spread of behavior.
Naturally one wants to know the most influential node in a network.

Calculating this is difficult as an exponential amount of conditional probabilities is necessary,
so simpler models or heuristics are needed.

\subsubsection{Deterministic Threshold models} % (fold)
\label{ssub:deterministic_threshold_models}
is a simple model that assumes nodes can be in two states.
Each node has a threshold $0 < p < 1$.
The node changes its state,
when the fraction of their neighbors that are in already in the new state exceeds their threshold.

\paragraph{Maximization} % (fold)
\label{par:maximization}
of influence by choosing the appropriate nodes to start with
is a NP-Hard problem.
It contains the NP-complete Vertex Cover problem as a special case.
% paragraph maximization (end)
% subsubsection deterministic_threshold_models (end)

\subsubsection{Probabilistic Influence Models} % (fold)
\label{ssub:probabilistic_influence_models}
In this model,
when a new node adopts the new behavior their is a given chance (hence the probability)
to infect one of its neighbors with the new behavior.
The probability can be determined by the weight of the edge,
connecting the nodes.
\paragraph{Maximization} % (fold)
\label{par:maximization}
is still a NP-Hard problem,
as it includes the NP-Complete set cover problem as a special case.
% paragraph maximization (end)
% subsubsection probabilistic_influence_models (end)

\subsubsection{Approximations} % (fold)
\label{ssub:approximations}
The Results can be approximated,
where the Greedy Hill-climbing is the most successful heuristic.

	\paragraph{Node Degree} % (fold)
	\label{par:node_degree}
	Selecting the node with the highest degree (e.g. number of friends)
	% paragraph node_degree (end)
	\paragraph{Distance Centrality} % (fold)
	\label{par:distance_centrality}
	Selecting the nodes with the lowest average distance to all other nodes.
	% paragraph distance_centrality (end)
	\paragraph{Greedy Hill-climbing} % (fold)
	\label{par:greedy_hill_climbing}
This algorithm chooses the node for which the objective grows the most.
% paragraph greedy_hill_climbing (end)
% subsubsection approximations (end)
% subsection social_influence_maximization (end)
% section lecture_6_information_cascade_&_influence_maximization (end)