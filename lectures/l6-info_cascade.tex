\section{Lecture 6 - Information Cascade \& Influence Maximization} % (fold)
\label{sec:lecture_6_information_cascade_&_influence_maximization}

This section focuses on the influence of information on the behavior
and the decisions of people.

\subsubsection{Information Cascades} % (fold)
\label{ssub:information_cascades}
Information cascades may occur when people make decisions sequentially,
like when later people watch the actions of earlier people
and then inferring something about what the earlier people know.
A cascade then develops when the people abandon their own information
in favor of inferences based on other peoples actions.
The later individuals imitate the behavior of other,
from inferences of limited information.

Milgram performed a second famous experiment,
where a group of people where standing at a lively street corner
and started watching into the sky.
Depending on the amount of people starring at the sky,
more people joined the group.
If 15 people were looking upwards,
45\% of the passersby would start looking up too.

\paragraph{Herding} % (fold)
\label{par:herding}
is the information cascade in the previous example.
It requires a decision to be made,
that can be taken out sequentially.
The decision is guided by some private information,
which cannot be observed by others.
They can only see the actions resulting from this private information.
% paragraph herding (end)
% subsubsection information_cascades (end)
% section lecture_6_information_cascade_&_influence_maximization (end)