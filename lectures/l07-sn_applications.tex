\section{Lecture 7 - Applications of Social Networks} % (fold)
\label{sec:lecture_7_applications_of_social_networks}

\subsection{Decentralized Search} % (fold)
\label{sub:decentralized_search}
The goal is to find the shortest path between a pair of nodes in a small world network.
As we have seen from Milgrams experiment,
a decentralized search where all nodes help is possible.

The principle is that node $s$ sending a message to node $t$,
without know the path to $t$.
Node $s$ only has local information
and does know the rough direction to node $t$.
It sends the message to a node it beliefs beeing closer to $t$.
The length of the searchpath is then the number of steps or hops till the message reaches $t$.

\subsubsection{General Network} % (fold)
\label{ssub:general_network}
	Depending on dimension it can either be a  ring or a grid.
	Each node is connected to its direct neighbors
	and has one long link.
	The probability of a link from $u$ to $v$ is then:
	
	\begin{equation}
		Pr\{u \rightarrow v\} ~ d(u, v)^{-q}
	\end{equation}
	Choosing the right value for the parameter $q$ depends on the network.
	It should not be too random and not to predictable.
	In experiements the ideal value is roughly at $1.75$.
% subsubsection general_network (end)
% subsection decentralized_search (end)

\subsection{Other Applications} % (fold)
\label{sub:other_applications}
Social networks were used in other cases to get information about epidemics or earthquakes.
Studies were conduct
and researchers were able to correlate the information from tweets to actual events.
They were able to learn about the spread of flu in UK
and that the amount of flu related tweets corresponds to the amount of people being sick from a flu.

In Japan and the USA researches could use tweets to get first hand information about earthquakes.
It took only a few seconds after the event for the first tweets related to show up.
The fast pace is key here.

This research is based on the semantic analysis of tweets,
which is feasible due to their shortness.
Tweets can be used as a sensory value.
Although there are some limitations,
as people in the endangered areas need to be using twitter.
If the distribution is not good enough,
the amount of information is to low to learn anything.
Also there might be incorrect geolocations associated with the tweets,
so that the information is incorrect.
Another problem is the instablility,
in case of an earthquake the networks service might be unreliable
and the system is prone to being attacked by malicious hackers.
% subsection other_applications (end)

% section lecture_7_applications_of_social_networks (end)