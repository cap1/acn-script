\documentclass{article}
\usepackage[utf8]{inputenc}
\usepackage{graphicx}
\usepackage{amsmath}
\usepackage{booktabs}
\usepackage{textcomp}
\usepackage{multirow}
\usepackage{color}

\author{Christian Müller}
\title{Advanced Computer Networks}

\begin{document}

\maketitle
\tableofcontents

\section{Lecture 1 Introduction}
The internet consists of millions of machines linked together using different technologies.
They are all connected via an a local ISP,
whom connects to a reginal ISP so that a hierarchical structure is created.

The network is typically devided in to Edges and Cores,
where at the edge one can find end systems, access networks and links.
Cores are circuit switching, packet switching, routers and network structure.
The dominant model for edges is a client/sever model.
Where one hosts acts as a server and another as client.
The server usually allows connections more than one client.
Although widely used for mail and web services,
the model ist not necessarily optimal in terms of scalabilty.

Another model is the peer-to-peer model,
which does only need minimal or even no dedicated servers.
Typical examples are file-sharing networks,
such as BitTorrent.

The Internet is organized in a hierarchical manner.
The users is connected via his local ISP to higher level ISP,
who interconnect (peer) with each other.
There are multiple ways to reach each endpoint in the the network.
For efficiency it is important to make sure that a packet takes an ideal way,
when traversing the network.
To ensure this,
it is important to know about the basic principles,
like the ISO/OSI model and about IP networks.
\end{document}