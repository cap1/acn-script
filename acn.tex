\documentclass{article}
\usepackage[utf8]{inputenc}
\usepackage{graphicx}
\usepackage{amsmath}
\usepackage{amssymb}
\usepackage{booktabs}
\usepackage{textcomp}
\usepackage{multirow}
\usepackage{color}

\author{Christian Müller}
\title{Advanced Computer Networks}
\setcounter{tocdepth}{2}

\begin{document}

\maketitle
\tableofcontents
\newpage

\section{Lecture 1 Introduction}
The internet consists of millions of machines linked together using different technologies.
They are all connected via an a local ISP,
whom connects to a reginal ISP so that a hierarchical structure is created.

The network is typically devided in to Edges and Cores,
where at the edge one can find end systems, access networks and links.
Cores are circuit switching, packet switching, routers and network structure.
The dominant model for edges is a client/sever model.
Where one hosts acts as a server and another as client.
The server usually allows connections more than one client.
Although widely used for mail and web services,
the model ist not necessarily optimal in terms of scalabilty.

Another model is the peer-to-peer model,
which does only need minimal or even no dedicated servers.
Typical examples are file-sharing networks,
such as BitTorrent.

The Internet is organized in a hierarchical manner.
The users is connected via his local ISP to higher level ISP,
who interconnect (peer) with each other.
There are multiple ways to reach each endpoint in the the network.
For efficiency it is important to make sure that a packet takes an ideal way,
when traversing the network.
To ensure this,
it is important to know about the basic principles,
like the ISO/OSI model and about IP networks.

\section{Lecture 2 - Introduction to P2P-Networks} % (fold)
\label{sec:lecture_2_introduction_to_p2p_networks}
Peer-to-Peer (P2P) Networks lack a central server so that all entities are equal in the network.

\subsection{Features} % (fold)
\label{sub:features}
\begin{description}
	\item[Decentralized] \hfill \\
		No central server or component, all Peers are equal
	\item[Self-organized] \hfill \\
		Dynamic behavior of nodes, free to join, part, quit
	\item[Overlay Network] \hfill \\
		Leverages the Internet and uses a flat architecture
	\item[Large Scale resource] \hfill \\
		Heterogeneous network with large possible number of nodes
	\item[Collaboration] \hfill \\
		Based on voluntary participation with global reach
	\item[Flexible] \hfill \\
		Resilient to attacks and anonymous
\end{description}
% subsection features (end)

\subsection{Problems with the Client/Server Model} % (fold)
\label{sub:problems_with_the_client_server_model}
The amount of Data the normal users transmit has been growing massively over last years.
Due to the increase in bandwidth,
users can also transmit these amounts of data.

\paragraph{Client/Server} % (fold)
\label{par:client_server}
is the classical model for the Internet.
A single server responds to the request of all users.
This server is also the problem of the model.
It may be easier to manage the single server,
but it is also the bottleneck and a single-point of failure.
If this server breaks down,
the service will not be available
and a single server just won't scale.
% paragraph client_server (end)

\paragraph{Replication} % (fold)
\label{par:replication}
solves some problems with the client/server model.
The server is mirrored so that it is not a single-point of failure anymore.
Although replication will still not scale well
and one can run into trouble due to high-workload on a single replica.
This single replica than has the same problems as a single server
as the workload is not distributed among the others.
Also maintenance might be a problem.
% paragraph replication (end)

\paragraph{Proxy \& CDN} % (fold)
\label{par:proxy_&_cdn}
can circumvent the problems with balancing the workload,
but still do not scale well in terms of bandwidth and CPU power.
And as they increase the complexity,
they maintenance gets more difficult.
% paragraph proxy_&_cdn (end)
% subsection problems_with_the_client_server_model (end)

\subsection{Peer 2 Peer} % (fold)
\label{sub:peer_2_peer}
Peer-to-Peer has some advantages over the classical client/server-model.
These systems work even better if more users are connected.
They are designed to be easy to use and to deploy,
to be fault tolerant.
Users can connect and leave dynamically and the quality of service is not altered.

\paragraph{Current State}
of P2P networks is that their overall share of network bandwidth used is declining.
This is due to the fact that the overall amount of bandwidth used is increasing,
while P2P traffic is not increasing.

Topics in research are the structure of P2P Networks,
regarding structured and unstructured P2P.
The security and the privacy of the users are also a research topic,
as well as the legal issues associated with P2P
Locating the the desired data in the P2P networks has always been an interesting thing,
as different types of networks solve this in different ways.

\paragraph{Structured \& Unstructured} % (fold)
\label{par:structured_unstructured}
are the main groups to sort P2P networks.
Unstructured networks are highly flexible,
dynamic and easy to maintain.
They base on the loosely connected nodes having no central point.
Due to this, it is hard to look information up.

Structured P2P networks have a fixed structure,
which enables an easy look up to find information.
% paragraph structured_unstructured (end)

\subsubsection{BitTorrent} % (fold)
\label{ssub:bittorrent}
BitTorrent is an unstructured P2P Network, that is very popular.
It accounted for 37\% of the Internet traffic in the German Internet in 2009.
The basic concepts of BitTorrent are \emph{chunking} and \emph{swarming}.
Where the files are split into smaller pieces or chunks,
so they can be downloaded in parallel and the order of download becomes irrelevant.
The chunks are then transfered between the nodes in a swarm,
who exchange the same content.

The basic components in a BitTorrent network are:
\begin{description}
	\item[Tracker] special server, tracking active peers and mapping file names to them
	\item[Peer] downloading as a leecher or sharing complete copy as seeder
	\item[.torrent] file containing the meta data of the data, ie. trackers IP and hashes of chunks
\end{description}

BitTorrent enforces a Tit-for-Tat policy,
so that users are encouraged to share bandwidth and data.
Also peers serve peers who that serve them,
so that free-riding is discouraged.
The chunk selection is done in a rares-first policy.
This maximizes the availability of each chunk.

BitTorrent is an efficient P2P protocol,
as it is fast, resilient, maximizes the availability of the data
and discourages free-riding.
Problems are that the tracker could be a single point of failure
and the lack of a search feature.
% subsubsection bittorrent (end)
% subsection peer_2_peer (end)

\subsection{Structured Peer 2 Peer Networks} % (fold)
\label{sub:structured_peer_2_peer_networks}
The common problem in P2P networks is how to implement the search for data in the network.
The requirements for the lookup of data in the network are as follows:
\begin{description}
	\item[Scalable] operating with millions of nodes
	\item[Self-organized] no central or external control
	\item[Load-distribution] all participants shall contributed
	\item[Fault-tolerant] robust against leaves and fails of nodes
	\item[Robustness] resilient against malicious nodes 
\end{description}

% subsection structured_peer_2_peer_networks (end)

\subsubsection{Distributed Hash Tables} % (fold)
\label{ssub:distributed_hash_tables}
The basic idea of Distributed Hash Tables (DHT) is to compute the hash value of the data
and the hash value of the IP addresses.
The content, or a content locator,
is then stored at the machine with the closest hash value.

The method has been introduced in 2001 in four Papers referring to the networks
\emph{CAN},  \emph{Chord}, \emph{Pastry} and \emph{Tapestry}.
\emph{BitTorrent}\ref{ssub:bittorrent} also makes use of Kadamelia DHT.

	\paragraph{Hash Functions} % (fold)
	\label{par:hash_functions}
	map an arbitrary input sequence to an output with a fixed length.
	A slight change in the input usually leads to a huge change in the output
	as the functions span the whole 2\textsuperscript{k} space of the hashlength \texttt{k}. 

	Typical has functions are \texttt{SHA-1} or \texttt{MD5},
	they usually differ in certain aspects like possibility of collisions and the hash value length.
	% paragraph hash_functions (end)
% subsubsection distributed_hash_tables (end)

\subsubsection{Chord} % (fold)
\label{ssub:chord}
	can serve as an example for the implementation of DHT.
	Given the identifier space is 2\textsuperscript{4} (0 .. 15).
	The nodes also have IDs in this space,
	which is not filled up.
	So the nodes just cover a subset of the identifiers available (2,5,6,11,14).
	The have a pointer to their successor,
	the node with a larger or equal ID in the clockwise direction.
	Each key, value pair of data is assigned the identifier of Hash(key).
	The item is then stored at its Successor(Hash(key)).

	The basic lookup can now be achieved in \textit{O(n)},
	as in the worst case one has to traverse one time over whole ring.
	This can be optimized to \textit{O(log n)} by adding a finger table to each node.
	This finger table contains fingers for $$succ(n+2^{i-1}$$ as shortcuts through the ring of nodes
	and thereby reduces the time to route to entries to \textit{$$log_2$$(n)}.

	Although each node just has a limited view of the network,
	the concept of numeric closeness solves the problems with routing and lookup.
	
	The lookup can be done iteratively,
	so that the requesting client keeps asking the next node where to continue looking.
	It can also be implemented recursively,
	so that the node querried first,
	will keep asking its successor and finally answer the request.
	Recursuve lookup allows each node to optimize its request
	and allows fast and efficient processing of the request.
	Iterative lookup allows the client to keep control
	and it can detect failures and localize them.

	\paragraph{Nodes join, leaving and failing} % (fold)
	\label{par:nodes_join_leaving_and_failing}
	are cases that need to be considered.
	When a new node joins in the ring,
	it needs to bootstrap its postiotn and find out about its neighbours.
	Also the routing needs to be altered.

	When a node joins and has found out about its own ID,
	it contacts tries to contact its successor to get the IDs of its successor and its predecessor.
	The nodes now change the responsibilty for documents in the range of the new node.
	To check if a node has left or has failed,
	successor are regulary pinged to see if they are still alive.
	This allows to treat fails as normal case.
	When a node is not reachable anymore,
	its successor gains responsibilty for the data of the leaving or failing node.
	% paragraph nodes_join_leaving_and_failing (end)

	\paragraph{Storage} % (fold)
	\label{par:storage}
	of data can be either direct or indirect.
	If it is impelmented directly,
	the data is stored on the nodes currently responsible for the data as defined by the ring structure.
	This leads to high overhead in communication
	and data transfer as the files need to be copied a lot.
	Especially if nodes are leaving and joining regularily.

	If only pointers to the actual data are stored on the nodes,
	the overhead is reduced and the load on the DHT is reduced.
	% paragraph storage (end)

	\paragraph{Successor sets} % (fold)
	\label{par:successor_sets}
	are similar to fingers.
	The nodes keep information about the successors and predecessors.
	They ping them regularly and replace them with the nearest node,
	which makes the ring less fragile as the nodes know about each other.

	With the \texttt{stablize()} function the nodes learn about each other
	and update their fingertables and successor sets.
	The lower nodes ask their successor who their predecessors are
	and inform them if they have changed.
	% paragraph successor_sets (end)
% subsubsection chord (end)

% section lecture_2_introduction_to_p2p_networks (end)
\newpage
\section{Lecture 3 - Online Social Networks} % (fold)
\label{sec:lecture_3_online_social_networks}

\subsection{Introduction} % (fold)
\label{sub:introduction}
Online Social Networks (OSN) are large online communities like Facebook and Twitter.
These are the basic terms in OSN:\\
\begin{description}
	\item[Social Network] \hfill \\
	A network of a set of individuals,
	connecting with each other on the basis of social relationships.
	\item[Entity] The basic unit of the network
	\item[Link] Interconnection between entities
	\item[Behavior \& Dynamics] \hfill \\
	The implicit consequences of the actions done by the entities.
	There is no isolation and small actions can lead to large effects.
\end{description}

Online Social Networks share some features with other networks.
At first glance they seem random,
bit they display signatures of order and \emph{self-organization}.
The characteristics are:\\
\begin{description}
	\item[Virtual] does not exist physically
	\item[Complex] consists of large number of nodes
	\item[Grouping] forms communities due to different similarities or interest
	\item[Dynamic] structure evolves in time
\end{description}
% subsection introduction (end)

\paragraph{The analysis of OSN} % (fold)
\label{par:the_analysis_of_osn}
can be done using different sets of methods.
Studying a network empirical, one can find the organization principles of the network.
With mathematical models on leverages probabilistic and graph theory
and there are also dedicated algorithms that help understanding  and analyzing networks.

The general targets one tries to achieve are:\\
\begin{itemize}
	\item patterns and statistical properties of network data
	\item desgin principles and models
	\item understand why OSNs are organized the way they are (prediction)
\end{itemize}
% paragraph the_analysis_of_osn (end)

\subsection{Modeling Social Networks} % (fold)
\label{sub:modeling_social_networks}
Networks can be represented as a mathematical graph.
A \emph{Node} is then an object on the network,
that can be connected via an \emph{Edge} to other nodes.

Depending on the graph the edges can be directed (Following on twitter)
or indirected (Friendship on Facebook).
A \emph{Path} then describess a sequence of interconnected nodes
and a \emph{Circle} is a Path where the first and the last nodes are the same.
The \emph{Connectivity} of a graph defines how well the nodes are interconnected.
A connected graph contains only pairs of nodes,
who have a path in between them.
If the graph is not connected,
it can be broken into several connected subgraphs.

A \emph{Connected Component} is a subset of nodes,
where every node in the subset has a path to every other node
and the subset ist not part of a larger subset with the same property.
Large complex networks often contain a \emph{giant component},
that comprises a significant fraction of all the nodes.

Social networks tend to cluster in groups,
called \emph{communities}.
The nodes inside of a community are stronger interconnected
and have fewer connections with nodes who are not part of the community.

The \emph{Betweenness} determines the amount of shortest paths,
passing through a given edge in the network.
% subsection modeling_social_networks (end)
% section lecture_3_online_social_networks (end)
\newpage
\section{Lecture 4 - Social Networks: Structure \& Properties} % (fold)
\label{sec:social_networks_structure_&_properties}

Networks can characterized in different ways.
Distinctive features are:\\
\begin{description}
	\item[Degree Distribution] How many Neighbors has a node?
	\item[Average Path Length] Whats the mean number of hops between nodes?
	\item[Network Diameter] How far apart are the nodes?
	\item[Distance - Shortest Path] How big is the network?
	\item[Clustering Coefficient] How close is a set of nodes connected with each other?
	\item[Community] Are subsets more closely connected?
\end{description}

\subsection{Network Properties} % (fold)
\label{sub:network_properties}
	The key Network properties are described in this Section,
	Degree Distribution, Path Length and Clustering Coefficient.

	\subsubsection{Degree Distribution} % (fold)
	\label{ssub:degree_distribution}
	The degree distribution $P(k)$ describes the probability of randomly given node,
	having the degree $k$.
	It can be calculated as described in Equation \ref{eq:degree_distribution}.

	\begin{equation}
	\label{eq:degree_distribution}
		P(k) = N_k / N
	\end{equation}

	Where $N_k$ is the number of nodes.
	% subsubsection degree_distribution (end)

	\subsubsection{Path Length} % (fold)
	\label{ssub:path_length}
		The Path Length subsumes several metrics for the network size.
		\paragraph{Distance} % (fold)
		\label{par:distance}
		describes the number of edges across the shortest path on the network.
		This also means that if two nods are not connected,
		their distance is infinite.
		% paragraph distance (end)

		\paragraph{Diameter} % (fold)
		\label{par:diameter}
		is the maximum distance in terms of edges between any nodes of the network.
		It is the opposite metric to the distance.
		% paragraph diameter (end)

		\paragraph{Average Path Length} % (fold)
		\label{par:average_path_length}
		is the combined measurement of the distance and the diameter,
		taking the average distance between all the nodes in the network.
		% paragraph average_path_length (end)
	% subsubsection path_length (end)

	\subsubsection{Clustering Coefficient} % (fold)
	\label{ssub:clustering_coefficient}
		This metric evaluated how the neighbors of a node are interconnected.
		For a given node $i$ with the degree $k$,
		assume the number of neighbors of $i$ is $e$,
		then the clustering coefficient of $i$ is described in equation \ref{eq:clustercoefficient}.
		\begin{equation}
		\label{eq:clustercoefficient}
		C_i - \frac{e}{k(k - 1)/2}
		\end{equation}

		The average Clustering Coefficient for network is then:
		\begin{equation}
		\label{eq:averagecc}
		C = \frac{1}{N} \sideset{}{_i^N}\sum C_i
		\end{equation}
	% subsubsection clustering_coefficient (end)

	\subsubsection{Degree Centrality} % (fold)
	\label{ssub:degree_centrality}
		This measurement determines how connected is a certain node inside the network.
		\begin{equation}
		\label{eq:degreecentrality}
			d_i(g) / (n-g)
		\end{equation}
	% subsubsection degree_centrality (end)

	\subsubsection{Closeness Centrality} % (fold)
	\label{ssub:closeness_centrality}
		Given the Closeness Centrality one can determine,
		how easily a node can reach other nodes in the network.
		\begin{equation}
		\label{eq:closenesscentrality}
			(n-1) / \sum_{j\neq i} l(i,j)
		\end{equation}
		Where $l(i,j)$ is the length of the shortest path between the nodes i and j.
	% subsubsection closeness_centrality (end)

	\subsubsection{Betweenness Centrality} % (fold)
	\label{ssub:betweenness_centrality}
		The Betweenness Centrality of a node gives a measurement how important a node is,
		in terms of connecting other nodes of the network.
		\begin{equation}
		\label{eq:betweennesscentrality}
			Ce_i^B (g) = \sum_{k\neq j:i\notin\{k,j\}} \frac{P_i(kj)/P(kj)}{(n-1)(n-2)/2}
		\end{equation}	
	% subsubsection betweenness_centrality (end)
% subsection network_properties (end)

\subsection{Metrics of typical networks} % (fold)
\label{sub:metrics_of_typical_networks}

	\subsubsection{Complete Graph} % (fold)
	\label{ssub:complete_graph}
	All nodes are interconnected.
		\begin{description}
			\item[Degree Distribution] $P(k) = N-1$
			\item[Path Length] Diameter = 1 --- Average Path Length = 1
			\item[Clustering Coefficient] C = 1 --- Average Clustering Coefficient = 1
		\end{description}
	% subsubsection complete_graph (end)

	\subsubsection*{Regular Latice} % (fold)
	\label{ssub:regular_latice}
	A Ring connecting only neighbors.
	Each node is also connected to its successors and its predecessors neighbor.
	\begin{description}
			\item[Degree Distribution] $P(k) = N-1$
			\item[Path Length] Diameter = 1 --- Average Path Length = 1
			\item[Clustering Coefficient] C = 1 --- Average Clustering Coefficient = 1
		\end{description}
	% subsubsection regular_latice (end)

	\subsubsection{Random Graph} % (fold)
	\label{ssub:random_graph}
	This depends on how the random graph is generated.
	Here we assume a binomial distribution.
		\begin{description}
			\item[Degree Distribution] $P(k) = \binom{n-1}{k}p^k(1-p)^{n-1-k}$
			\item[Average Path Length] $O(\log n)$
			\item[Clustering Coefficient] $C = p = \bar{k}/n$
		\end{description}
	% subsubsection random_graph (end)
% subsection metrics_of_typical_networks (end)

\subsection{Power Law} % (fold)
\label{sub:power_law}
When observing the popularity of nodes in a network,
one can see that there is an imbalance.
Some nodes have far higher degrees and receive more connections than others.
A typical example is that 20\% of the websites,
receive 80\% of the traffic.
This effect can be observed in many other networks like twitter,
where celebrities have a huge amount of followers,
or the imbalance of money distribution,
where only a few very rich people own a large amount.

\begin{itemize}
	\item The fraction of telephone numbers that receive calls per day is roughly equivalent to $1/k^2$
	\item The fraction of books bought is roughly proportional to $1/k^3$
	\item The fraction of scientific paper cited is roughly proportional to $1/k^3$
\end{itemize}

The degree distribution is not a Gaussian one,
but a \emph{Power Law distribution}.

	\subsubsection{Features} % (fold)
	\label{ssub:features}
	% subsubsection features (end)
% subsection power_law (end)

\subsection{Small World Phenomenon} % (fold)
\label{sub:small_world_phenomenon}
This Phenomenon the fact,
that most networks are quite densely connected
and that the distance is much lower than expected.
Milgram found out about this in his famous experiment in 1967,
where it took the letters an average of 6.2 hops to reach the target.
In todays network the number of hops 
and thereby the diameter are even lower.
On twitter it was at 4.67 and 4.74 on Facebook both in 2011.

The simple explanation is quite simple.
If one supposes that each person knows 100 people on a first name basis,
due to the exponential effects one can reach $100^5 = 10 Billion$ people in the 5\textsuperscript{th} step.
As the world only inhabits about 7 Billion people,
the chances of finding a connection is quite large.

\subsubsection{Features of Small World networks} % (fold)
\label{ssub:features_of_small_world_networks}
	Small World Networks are defined,
	so that most nodes are not neighbors,
	but most nodes can be reached by a small number of hops.

	It hast a \emph{high Clustering Coefficient}
	and a \emph{low Diameter}.
	The Diameter is almost equal to a random network of the same size.
% subsubsection features_of_small_world_networks (end)
% subsection small_world_phenomenon (end)

% section social_networks_structure_&_properties (end)
\newpage
\section{Lecture 5 - Network Models} % (fold)
\label{sec:lecture_5_network_models}
As social networks play an important role on its inhabitants.
They influence the flow of information,
the spread of disease,
the trade of goods and services as well as which languages we speak and what we vote.\\
With network analysis technologies we can try to understand,
how their structure influences us
and which structures are likely to emerge in the society.

\paragraph{Marriage Network in Florentine} % (fold)
\label{par:marriage_network_in_florentine}
	An example can be the network of marriages in Florentine in the fifteenth century.
	Through network analysis one can find hints for the shift of power from the 
	Strozzi family to the Medici family.
	The Degree Centrality (\ref{ssub:degree_centrality}) for the Strozzi family in the network
	was only 4/15, but for the Medicis it was 6/15.
	The Betweenness Centrality of the Strozzis was $0.103$,
	which is very low compared to the $05.22$ of the Medicis,
	as more than half of the shortest paths between influential families went through the Medicis.
	This not only illustrates the importance of the families,
	but shows that knowledge about network structure gives insights into political and economic structures.
% paragraph marriage_network_in_florentine (end)

\subsection[Random Network]{Random Network - Erd\"os-Renyi\\} % (fold)
\label{ssub:erdoes-renyi}
These networks are generated in a random process,
whom origins with fixed set of nodes $N = {1,2,..,n}$.
Then links between the nodes are added with a given probability $p (0 < p < 1)$.\\
For a given network with $m$ links,
the probability that it is formed is:

\begin{equation}
\label{eq:erdoesrenyiprob}
	p^m (1 - p)^{\frac{n(n-1)}{2}-m}
\end{equation}
The Degree Distribution (\ref{ssub:degree_distribution})
that any given node $i$ has exactly $d$ links is then

\begin{equation}
\label{eq:erdoesrenyidegreedist}
	Pr(d) = \binom{n-1}{d} p^d(1-p)^{n-1-d}
\end{equation}
For large $n$ and small $p$,
this binomial expression is approximated by a Poisson distribution

\begin{equation}
\label{eq:erdoesrenyipoisson}
	Pr(d) = \frac{e^{-(n-1)p}((n-1)p)^d}{d!}
\end{equation}
Such networks are called \emph{Poisson Random Networks}.
The connectivity of the graph can be calculated easily with Equation \ref{eq:erdoesrenyipoisson}.

\begin{equation}
	Pr(0) = e^{-(n-1)p}
\end{equation}
One can further solve this and when:

\begin{equation}
	(n-1)p= log (n)
\end{equation}
is larger than $p$ the graph will break in several components.
If $p > (n-1)p= log (n)$ the graph is connected with a high probability.
% subsubsection erdoesrenyi (end)

\subsection[Small World Network]{Small World Network - Watts-Strogatz} % (fold)
\label{sub:watts_strogatz}
This Model tries to generate a network with a low diameter (\ref{par:diameter})
and a high clustering coefficient (\ref{ssub:clustering_coefficient}).
The generation starts with a regular lattice
and continuously rewires the network with a given probability $p | 0<p<1$.
With $p = 0$, a regular network will be generated
and with $p = 1$ a random one will emerge.
The randomization takes place by lowering the average path length (\ref{par:average_path_length}).

Networks created with \emph{Small World Model} have a high clustering coefficient
and a low diameter.
% subsection watts_strogatz (end)

\subsection{Scale-free Network} % (fold)
\label{sub:scale_free_network}
Scale-free Networks are networks whose degree distribution follows a Power law.
Several models are proposed to generate these networks randomly.
The simple \emph{Rich get richer model},
which will be explained here,
the preferential attachment by Barabasi 
and the Fitness Model by Biancomi and by Kong.

\paragraph{The Rich get richer model} % (fold)
\label{par:the_rich_get_richer_model}
follows the Power law.
In the process of creation new nodes are added to the network in each step.
With a given probability this node.

In an iterative process the network is created.
When a node $j$ is added to the network,
it produces a link either to a node $i$ choosen uniformly from the set of previously created nodes
with the probabilty ($0 < p < 1$) or \\
with probability ($1-p$) the node $j$ chooses a node $i$ uniformly at random
and links to a node $i$ points to.
% paragraph the_rich_get_richer_model (end)
% subsection scale_free_network (end)


% section lecture_5_network_models (end)
\newpage
\section{Lecture 6 - Information Cascade \& Influence Maximization} % (fold)
\label{sec:lecture_6_information_cascade_&_influence_maximization}

This section focuses on the influence of information on the behavior
and the decisions of people.

\subsection{Information Cascades} % (fold)
\label{sub:information_cascades}
Information cascades may occur when people make decisions sequentially,
like when later people watch the actions of earlier people
and then inferring something about what the earlier people know.
A cascade then develops when the people abandon their own information
in favor of inferences based on other peoples actions.
The later individuals imitate the behavior of other,
from inferences of limited information.

Milgram performed a second famous experiment,
where a group of people where standing at a lively street corner
and started watching into the sky.
Depending on the amount of people starring at the sky,
more people joined the group.
If 15 people were looking upwards,
45\% of the passersby would start looking up too.

\paragraph{Herding} % (fold)
\label{par:herding}
is the information cascade in the previous example.
It requires a decision to be made,
that can be taken out sequentially.
The decision is guided by some private information,
which cannot be observed by others.
They can only see the actions resulting from this private information.
% paragraph herding (end)

\paragraph{Effects} % (fold)
\label{par:effects}
Information Cascades and Rich-get-richer-models explain some parts of social behavior.
This can be observed in Rumors,
adaption of new technologies in a peer group,
fashion and even in stock markets.
% paragraph effects (end)
% subsection information_cascades (end)

\subsection{Social Influence Maximization} % (fold)
\label{sub:social_influence_maximization}
Networks consist in this context of nodes (people)
and edges (social interactions, ties).

The behavior of people is influenced by the behavior of other people.
Understanding this influence is interesting for several aspects
like rumours, collective dynamics and spread of behavior.
Naturally one wants to know the most influential node in a network.

Calculating this is difficult as an exponential amount of conditional probabilities is necessary,
so simpler models or heuristics are needed.

\subsubsection{Deterministic Threshold models} % (fold)
\label{ssub:deterministic_threshold_models}
is a simple model that assumes nodes can be in two states.
Each node has a threshold $0 < p < 1$.
The node changes its state,
when the fraction of their neighbors that are in already in the new state exceeds their threshold.

\paragraph{Maximization} % (fold)
\label{par:maximization}
of influence by choosing the appropriate nodes to start with
is a NP-Hard problem.
It contains the NP-complete Vertex Cover problem as a special case.
% paragraph maximization (end)
% subsubsection deterministic_threshold_models (end)

\subsubsection{Probabilistic Influence Models} % (fold)
\label{ssub:probabilistic_influence_models}
In this model,
when a new node adopts the new behavior their is a given chance (hence the probability)
to infect one of its neighbors with the new behavior.
The probability can be determined by the weight of the edge,
connecting the nodes.
\paragraph{Maximization} % (fold)
\label{par:maximization}
is still a NP-Hard problem,
as it includes the NP-Complete set cover problem as a special case.
% paragraph maximization (end)
% subsubsection probabilistic_influence_models (end)

\subsubsection{Approximations} % (fold)
\label{ssub:approximations}
The Results can be approximated,
where the Greedy Hill-climbing is the most successful heuristic.

	\paragraph{Node Degree} % (fold)
	\label{par:node_degree}
	Selecting the node with the highest degree (e.g. number of friends)
	% paragraph node_degree (end)
	\paragraph{Distance Centrality} % (fold)
	\label{par:distance_centrality}
	Selecting the nodes with the lowest average distance to all other nodes.
	% paragraph distance_centrality (end)
	\paragraph{Greedy Hill-climbing} % (fold)
	\label{par:greedy_hill_climbing}
This algorithm chooses the node for which the objective grows the most.
% paragraph greedy_hill_climbing (end)
% subsubsection approximations (end)
% subsection social_influence_maximization (end)
% section lecture_6_information_cascade_&_influence_maximization (end)
\newpage
\section{Lecture 7 - Applications of Social Networks} % (fold)
\label{sec:lecture_7_applications_of_social_networks}

\subsection{Decentralized Search} % (fold)
\label{sub:decentralized_search}
The goal is to find the shortest path between a pair of nodes in a small world network.
As we have seen from Milgrams experiment,
a decentralized search where all nodes help is possible.

The principle is that node $s$ sending a message to node $t$,
without know the path to $t$.
Node $s$ only has local information
and does know the rough direction to node $t$.
It sends the message to a node it beliefs beeing closer to $t$.
The length of the searchpath is then the number of steps or hops till the message reaches $t$.

\subsubsection{General Network} % (fold)
\label{ssub:general_network}
	Depending on dimension it can either be a  ring or a grid.
	Each node is connected to its direct neighbors
	and has one long link.
	The probability of a link from $u$ to $v$ is then:
	
	\begin{equation}
		Pr\{u \rightarrow v\} ~ d(u, v)^{-q}
	\end{equation}
	Choosing the right value for the parameter $q$ depends on the network.
	It should not be too random and not to predictable.
	In experiements the ideal value is roughly at $1.75$.
% subsubsection general_network (end)
% subsection decentralized_search (end)

\subsection{Other Applications} % (fold)
\label{sub:other_applications}
Social networks were used in other cases to get information about epidemics or earthquakes.
Studies were conduct
and researchers were able to correlate the information from tweets to actual events.
They were able to learn about the spread of flu in UK
and that the amount of flu related tweets corresponds to the amount of people being sick from a flu.

In Japan and the USA researches could use tweets to get first hand information about earthquakes.
It took only a few seconds after the event for the first tweets related to show up.
The fast pace is key here.

This research is based on the semantic analysis of tweets,
which is feasible due to their shortness.
Tweets can be used as a sensory value.
Although there are some limitations,
as people in the endangered areas need to be using twitter.
If the distribution is not good enough,
the amount of information is to low to learn anything.
Also there might be incorrect geolocations associated with the tweets,
so that the information is incorrect.
Another problem is the instablility,
in case of an earthquake the networks service might be unreliable
and the system is prone to being attacked by malicious hackers.
% subsection other_applications (end)

% section lecture_7_applications_of_social_networks (end)
\newpage
\section{Lecture 9 - Scaling of OSN} % (fold)
\label{sec:lecture_9_scaling_of_osn}

% section lecture_9_scaling_of_osn (end)

\end{document}