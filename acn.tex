\documentclass{article}
\usepackage[utf8]{inputenc}
\usepackage{graphicx}
\usepackage{amsmath}
\usepackage{amssymb}
\usepackage{booktabs}
\usepackage{textcomp}
\usepackage{multirow}
\usepackage{color}

\author{Christian Müller}
\title{Advanced Computer Networks}
\setcounter{tocdepth}{2}

\begin{document}

\maketitle
\tableofcontents
\newpage

\section{Lecture 1 Introduction}
The internet consists of millions of machines linked together using different technologies.
They are all connected via an a local ISP,
whom connects to a reginal ISP so that a hierarchical structure is created.

The network is typically devided in to Edges and Cores,
where at the edge one can find end systems, access networks and links.
Cores are circuit switching, packet switching, routers and network structure.
The dominant model for edges is a client/sever model.
Where one hosts acts as a server and another as client.
The server usually allows connections more than one client.
Although widely used for mail and web services,
the model ist not necessarily optimal in terms of scalabilty.

Another model is the peer-to-peer model,
which does only need minimal or even no dedicated servers.
Typical examples are file-sharing networks,
such as BitTorrent.

The Internet is organized in a hierarchical manner.
The users is connected via his local ISP to higher level ISP,
who interconnect (peer) with each other.
There are multiple ways to reach each endpoint in the the network.
For efficiency it is important to make sure that a packet takes an ideal way,
when traversing the network.
To ensure this,
it is important to know about the basic principles,
like the ISO/OSI model and about IP networks.

\input{./lectures/l02-p2p}
\newpage
\input{./lectures/l03-osn}
\newpage
\input{./lectures/l04-osn_struct-prop}
\newpage
\input{./lectures/l05-network_models}
\newpage
\section{Lecture 6 - Information Cascade \& Influence Maximization} % (fold)
\label{sec:lecture_6_information_cascade_&_influence_maximization}

This section focuses on the influence of information on the behavior
and the decisions of people.

\subsection{Information Cascades} % (fold)
\label{sub:information_cascades}
Information cascades may occur when people make decisions sequentially,
like when later people watch the actions of earlier people
and then inferring something about what the earlier people know.
A cascade then develops when the people abandon their own information
in favor of inferences based on other peoples actions.
The later individuals imitate the behavior of other,
from inferences of limited information.

Milgram performed a second famous experiment,
where a group of people where standing at a lively street corner
and started watching into the sky.
Depending on the amount of people starring at the sky,
more people joined the group.
If 15 people were looking upwards,
45\% of the passersby would start looking up too.

\paragraph{Herding} % (fold)
\label{par:herding}
is the information cascade in the previous example.
It requires a decision to be made,
that can be taken out sequentially.
The decision is guided by some private information,
which cannot be observed by others.
They can only see the actions resulting from this private information.
% paragraph herding (end)

\paragraph{Effects} % (fold)
\label{par:effects}
Information Cascades and Rich-get-richer-models explain some parts of social behavior.
This can be observed in Rumors,
adaption of new technologies in a peer group,
fashion and even in stock markets.
% paragraph effects (end)
% subsection information_cascades (end)

\subsection{Social Influence Maximization} % (fold)
\label{sub:social_influence_maximization}
Networks consist in this context of nodes (people)
and edges (social interactions, ties).

The behavior of people is influenced by the behavior of other people.
Understanding this influence is interesting for several aspects
like rumours, collective dynamics and spread of behavior.
Naturally one wants to know the most influential node in a network.

Calculating this is difficult as an exponential amount of conditional probabilities is necessary,
so simpler models or heuristics are needed.

\subsubsection{Deterministic Threshold models} % (fold)
\label{ssub:deterministic_threshold_models}
is a simple model that assumes nodes can be in two states.
Each node has a threshold $0 < p < 1$.
The node changes its state,
when the fraction of their neighbors that are in already in the new state exceeds their threshold.

\paragraph{Maximization} % (fold)
\label{par:maximization}
of influence by choosing the appropriate nodes to start with
is a NP-Hard problem.
It contains the NP-complete Vertex Cover problem as a special case.
% paragraph maximization (end)
% subsubsection deterministic_threshold_models (end)

\subsubsection{Probabilistic Influence Models} % (fold)
\label{ssub:probabilistic_influence_models}
In this model,
when a new node adopts the new behavior their is a given chance (hence the probability)
to infect one of its neighbors with the new behavior.
The probability can be determined by the weight of the edge,
connecting the nodes.
\paragraph{Maximization} % (fold)
\label{par:maximization}
is still a NP-Hard problem,
as it includes the NP-Complete set cover problem as a special case.
% paragraph maximization (end)
% subsubsection probabilistic_influence_models (end)

\subsubsection{Approximations} % (fold)
\label{ssub:approximations}
The Results can be approximated,
where the Greedy Hill-climbing is the most successful heuristic.

	\paragraph{Node Degree} % (fold)
	\label{par:node_degree}
	Selecting the node with the highest degree (e.g. number of friends)
	% paragraph node_degree (end)
	\paragraph{Distance Centrality} % (fold)
	\label{par:distance_centrality}
	Selecting the nodes with the lowest average distance to all other nodes.
	% paragraph distance_centrality (end)
	\paragraph{Greedy Hill-climbing} % (fold)
	\label{par:greedy_hill_climbing}
This algorithm chooses the node for which the objective grows the most.
% paragraph greedy_hill_climbing (end)
% subsubsection approximations (end)
% subsection social_influence_maximization (end)
% section lecture_6_information_cascade_&_influence_maximization (end)
\newpage
\input{./lectures/l07-sn_applications}
\newpage
\section{Lecture 9 - Scaling of OSN} % (fold)
\label{sec:lecture_9_scaling_of_osn}

\subsection{Introduction to OSN Scaling}
As Online Social Networks are extremly popular,
their demand for a scalable ifrastructure to support their grwoth is natural.
Cloud COmputing can provide such scaling,
but it leads to some problems that need to be deatl with.

Conventional scaling approaches are horizontal or vertical.
Where \emph{vertical} stands for the the upgrading the existing hardware.
This can be expensive and be techincally infeasible.
\emph{Horizontal} scaling means deploying a larger number of servers
and distributing the workload amongst them.
Although this is only suitable for stateless frontend servers,
on backend storage servers the data must be partitioned into disjoint components.

Bot approaches are not applicable to OSNs,
as the large amaout of data rules out vertical sacling 
and the large amount of interconnect makes horizontal scaling infeasible.
This refers to the fact that in an OSN,
ost operations are between users and their neighbors.
Their data is usually placed on multiple servers
and the mu 'multi-get' operations genrate a huge amount of traffic
and unpredictable responsetimes.

\subsection{Social Paritioning and Replication} % (fold)
\label{sub:social_paritioning_and_replication}
A novel soulution for these problmes could be \emph{SPAR} (Social Partitioning And Replication).
Its approach is to have a one-hop replication
so that all neighbors of users are replicated on the same server.
This leads to a 'Social Locality'.\\
The requirements for SPAR are:
	\begin{itemize}
		\item Maintain local semantics
		\item Balance loads
		\item Be resilient to machine failures
		\item Be amenable to online operations
		\item Be stable
		\item Minimize the replication overhead
	\end{itemize}
% subsection social_paritioning_and_replication (end)

\subsection{OSN in the Cloud} % (fold)
\label{sub:osn_in_the_cloud}
OSNs require features that can be delivered very well with cloud computing.
The networks need to be available in a reliable fashion
on a global scale.
They also require high amounts of flexibility in the amount of workload they can do.

Cloud computing can provide the availability, flexibility and scalability required by OSNs.
Due to the 'pay-as-you-go' plans the cost do not have to paid in front.
The cloud allows to serve customers,
but keep the cost at a low level.
Although the Quality of Service (QoS) must be maintained.
% subsection osn_in_the_cloud (end)
% section lecture_9_scaling_of_osn (end)
\newpage
\section{Lecture 10 - Content Centric Networks I} % (fold)
\label{sec:lecture_10_content_centric_networks_1}

% section lecture_10_content_centric_networks_1 (end)

\end{document}