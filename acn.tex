\documentclass{article}
\usepackage[utf8]{inputenc}
\usepackage{graphicx}
\usepackage{amsmath}
\usepackage{amssymb}
\usepackage{booktabs}
\usepackage{textcomp}
\usepackage{multirow}
\usepackage{color}

\author{Christian Müller}
\title{Advanced Computer Networks}
\setcounter{tocdepth}{2}

\begin{document}

\maketitle
\tableofcontents
\newpage

\section{Lecture 1 Introduction}
The internet consists of millions of machines linked together using different technologies.
They are all connected via an a local ISP,
whom connects to a reginal ISP so that a hierarchical structure is created.

The network is typically devided in to Edges and Cores,
where at the edge one can find end systems, access networks and links.
Cores are circuit switching, packet switching, routers and network structure.
The dominant model for edges is a client/sever model.
Where one hosts acts as a server and another as client.
The server usually allows connections more than one client.
Although widely used for mail and web services,
the model ist not necessarily optimal in terms of scalabilty.

Another model is the peer-to-peer model,
which does only need minimal or even no dedicated servers.
Typical examples are file-sharing networks,
such as BitTorrent.

The Internet is organized in a hierarchical manner.
The users is connected via his local ISP to higher level ISP,
who interconnect (peer) with each other.
There are multiple ways to reach each endpoint in the the network.
For efficiency it is important to make sure that a packet takes an ideal way,
when traversing the network.
To ensure this,
it is important to know about the basic principles,
like the ISO/OSI model and about IP networks.

\section{Lecture 2 - Introduction to P2P-Networks} % (fold)
\label{sec:lecture_2_introduction_to_p2p_networks}
Peer-to-Peer (P2P) Networks lack a central server so that all entities are equal in the network.

\subsection{Features} % (fold)
\label{sub:features}
\begin{description}
	\item[Decentralized] \hfill \\
		No central server or component, all Peers are equal
	\item[Self-organized] \hfill \\
		Dynamic behavior of nodes, free to join, part, quit
	\item[Overlay Network] \hfill \\
		Leverages the Internet and uses a flat architecture
	\item[Large Scale resource] \hfill \\
		Heterogeneous network with large possible number of nodes
	\item[Collaboration] \hfill \\
		Based on voluntary participation with global reach
	\item[Flexible] \hfill \\
		Resilient to attacks and anonymous
\end{description}
% subsection features (end)

\subsection{Problems with the Client/Server Model} % (fold)
\label{sub:problems_with_the_client_server_model}
The amount of Data the normal users transmit has been growing massively over last years.
Due to the increase in bandwidth,
users can also transmit these amounts of data.

\paragraph{Client/Server} % (fold)
\label{par:client_server}
is the classical model for the Internet.
A single server responds to the request of all users.
This server is also the problem of the model.
It may be easier to manage the single server,
but it is also the bottleneck and a single-point of failure.
If this server breaks down,
the service will not be available
and a single server just won't scale.
% paragraph client_server (end)

\paragraph{Replication} % (fold)
\label{par:replication}
solves some problems with the client/server model.
The server is mirrored so that it is not a single-point of failure anymore.
Although replication will still not scale well
and one can run into trouble due to high-workload on a single replica.
This single replica than has the same problems as a single server
as the workload is not distributed among the others.
Also maintenance might be a problem.
% paragraph replication (end)

\paragraph{Proxy \& CDN} % (fold)
\label{par:proxy_&_cdn}
can circumvent the problems with balacing the workload,
but still do not scale well in terms of bandwidth and CPU power.
And as they increase the complexity,
they maintenance gets more difficutlt.
% paragraph proxy_&_cdn (end)
% subsection problems_with_the_client_server_model (end)

\subsection{Peer 2 Peer} % (fold)
\label{sub:peer_2_peer}
Peer-to-Peer has some advantages over the classical client/server-model.
These systems work even better if more users are connected.
They are designed to be easy to use and to deploy,
to be fault tolerant.
Users can connect and leave dynamically and the quality of service is not altered.

\paragraph{Current State}
of P2P networks is that their overall share of network bandwidth used is declining.
This is due to the fact that the overall amount of bandwith used is increasing,
while P2P traffic is not increasing.
% subsection peer_2_peer (end)

% section lecture_2_introduction_to_p2p_networks (end)

\newpage
\section{Lecture 3 - Online Social Networks} % (fold)
\label{sec:lecture_3_online_social_networks}

\subsection{Introduction} % (fold)
\label{sub:introduction}
Online Social Networks (OSN) are large online communities like Facebook and Twitter.
These are the basic terms in OSN:\\
\begin{description}
	\item[Social Network] \hfill \\
	A network of a set of individuals,
	connecting with each other on the basis of social relationships.
	\item[Entity] The basic unit of the network
	\item[Link] Interconnection between entities
	\item[Behavior \& Dynamics] \hfill \\
	The implicit consequences of the actions done by the entities.
	There is no isolation and small actions can lead to large effects.
\end{description}

Online Social Networks share some features with other networks.
At first glance they seem random,
bit they display signatures of order and \emph{self-organization}.
The characteristics are:\\
\begin{description}
	\item[Virtual] does not exist physically
	\item[Complex] consists of large number of nodes
	\item[Grouping] forms communities due to different similarities or interest
	\item[Dynamic] structure evolves in time
\end{description}
% subsection introduction (end)

\paragraph{The analysis of OSN} % (fold)
\label{par:the_analysis_of_osn}
can be done using different sets of methods.
Studying a network empirical, one can find the organization principles of the network.
With mathematical models on leverages probabilistic and graph theory
and there are also dedicated algorithms that help understanding  and analyzing networks.

The general targets one tries to achieve are:\\
\begin{itemize}
	\item patterns and statistical properties of network data
	\item desgin principles and models
	\item understand why OSNs are organized the way they are (prediction)
\end{itemize}
% paragraph the_analysis_of_osn (end)

\subsection{Modeling Social Networks} % (fold)
\label{sub:modeling_social_networks}
Networks can be represented as a mathematical graph.
A \emph{Node} is then an object on the network,
that can be connected via an \emph{Edge} to other nodes.

Depending on the graph the edges can be directed (Following on twitter)
or indirected (Friendship on Facebook).
A \emph{Path} then describess a sequence of interconnected nodes
and a \emph{Circle} is a Path where the first and the last nodes are the same.
The \emph{Connectivity} of a graph defines how well the nodes are interconnected.
A connected graph contains only pairs of nodes,
who have a path in between them.
If the graph is not connected,
it can be broken into several connected subgraphs.
A \emph{Connected Component} is a subset of nodes,
where every node in the subset has a path to every other node
and the subset ist not part of a larger subset with the same property.
% subsection modeling_social_networks (end)
% subsection subsection_name (end)

% section lecture_3_online_social_networks (end)

\newpage
\section{Lecture 4 - Social Networks: Structure \& Properties} % (fold)
\label{sec:social_networks_structure_&_properties}

Networks can characterized in different ways.
Distinctive features are:\\
\begin{description}
	\item[Degree Distribution] How many Neighbors has a node?
	\item[Average Path Length] Whats the mean number of hops between nodes?
	\item[Network Diameter] How far apart are the nodes?
	\item[Distance - Shortest Path] How big is the network?
	\item[Clustering Coefficient] How close is a set of nodes connected with each other?
	\item[Community] Are subsets more closely connected?
\end{description}

\subsection{Network Properties} % (fold)
\label{sub:network_properties}
	The key Network properties are described in this Section,
	Degree Distribution, Path Length and Clustering Coefficient.

	\subsubsection{Degree Distribution} % (fold)
	\label{ssub:degree_distribution}
	The degree distribution $P(k)$ describes the probability of randomly given node,
	having the degree $k$.
	It can be calculated as described in Equation \ref{eq:degree_distribution}.

	\begin{equation}
	\label{eq:degree_distribution}
		P(k) = N_k / N
	\end{equation}

	Where $N_k$ is the number of nodes.
	% subsubsection degree_distribution (end)

	\subsubsection{Path Length} % (fold)
	\label{ssub:path_length}
		The Path Length subsumes several metrics for the network size.
		\paragraph{Distance} % (fold)
		\label{par:distance}
		describes the number of edges across the shortest path on the network.
		This also means that if two nods are not connected,
		their distance is infinite.
		% paragraph distance (end)

		\paragraph{Diameter} % (fold)
		\label{par:diameter}
		is the maximum distance in terms of edges between any nodes of the network.
		It is the opposite metric to the distance.
		% paragraph diameter (end)

		\paragraph{Average Path Length} % (fold)
		\label{par:average_path_length}
		is the combined measurement of the distance and the diameter,
		taking the average distance between all the nodes in the network.
		% paragraph average_path_length (end)
	% subsubsection path_length (end)

	\subsubsection{Clustering Coefficient} % (fold)
	\label{ssub:clustering_coefficient}
		This metric evaluated how the neighbors of a node are interconnected.
		For a given node $i$ with the degree $k$,
		assume the number of neighbors of $i$ is $e$,
		then the clustering coefficient of $i$ is described in equation \ref{eq:clustercoefficient}.
		\begin{equation}
		\label{eq:clustercoefficient}
		C_i - \frac{e}{k(k - 1)/2}
		\end{equation}

		The average Clustering Coefficient for network is then:
		\begin{equation}
		\label{eq:averagecc}
		C = \frac{1}{N} \sideset{}{_i^N}\sum C_i
		\end{equation}
	% subsubsection clustering_coefficient (end)

\subsection{Metrics of typical networks} % (fold)
\label{sub:metrics_of_typical_networks}

	\subsubsection{Complete Graph} % (fold)
	\label{ssub:complete_graph}
	All nodes are interconnected.
		\begin{description}
			\item[Degree Distribution] $P(k) = N-1$
			\item[Path Length] Diameter = 1 --- Average Path Length = 1
			\item[Clustering Coefficient] C = 1 --- Average Clustering Coefficient = 1
		\end{description}
	% subsubsection complete_graph (end)

	\subsubsection*{Regular Latice} % (fold)
	\label{ssub:regular_latice}
	A Ring connecting only neighbors.
	Each node is also connected to its successors and its predecessors neighbor.
	\begin{description}
			\item[Degree Distribution] $P(k) = N-1$
			\item[Path Length] Diameter = 1 --- Average Path Length = 1
			\item[Clustering Coefficient] C = 1 --- Average Clustering Coefficient = 1
		\end{description}
	% subsubsection regular_latice (end)

	\subsubsection{Random Graph} % (fold)
	\label{ssub:random_graph}
	This depends on how the random graph is generated.
	Here we assume a binomial distribution.
		\begin{description}
			\item[Degree Distribution] $P(k) = \binom{n-1}{k}p^k(1-p)^{n-1-k}$
			\item[Average Path Length] $O(\log n)$
			\item[Clustering Coefficient] $C = p = \bar{k}/n$
		\end{description}
	% subsubsection random_graph (end)
% subsection metrics_of_typical_networks (end)

% subsection network_properties (end)
% section social_networks_structure_&_properties (end)

\end{document}